\documentclass[11pt]{article}
\usepackage{xspace,epsfig,amsmath,amsthm,amssymb,fullpage}
\topmargin   0pt
\marginparwidth 0pt
\oddsidemargin  0pt
\evensidemargin 0pt
\marginparsep 0pt
\textwidth   6.5in
\textheight  9.2in

\usepackage{tikz}
\usepackage{tikz-qtree,tikz-qtree-compat} % for parse trees

\usepackage{latexsym}
\usepackage{amssymb}
\usepackage{amsmath}
\usepackage{amsthm}
\usepackage{amsfonts}
\usepackage{bm} 
% \usepackage{prooftree}
\usepackage{flagderiv}
\usepackage{logicproof}
\usepackage{bussproofs}
\usepackage{hyperref}
\usepackage{color}
\usepackage{listings}
\usepackage{synttree} 
\usepackage{pdflscape}
\usepackage{enumerate}   

\definecolor{keywordcolor}{rgb}{0.7, 0.1, 0.1}   % red
\definecolor{tacticcolor}{rgb}{0.0, 0.1, 0.6}    % blue
\definecolor{commentcolor}{rgb}{0.4, 0.4, 0.4}   % grey
\definecolor{symbolcolor}{rgb}{0.0, 0.1, 0.6}    % blue
\definecolor{sortcolor}{rgb}{0.1, 0.5, 0.1}      % green
\definecolor{attributecolor}{rgb}{0.7, 0.1, 0.1} % red
\newcommand*{\defeq}{\stackrel{\text{def}}{=}}

\def\lstlanguagefiles{lstlean.tex}
% set default language

\hypersetup{colorlinks=true,linkcolor=blue,urlcolor=blue}
% WARNING:
% Do NOT use package `bussproofs' and package `prooftree' at the same time,
% \begin{prooftree} ... \end{prooftree} is an environment defined in the
% package  `bussproofs', which conflicts with the name of the package
% `prooftree'.

\newcommand{\defn}{\overset{\text{\scriptsize def}}{=}}
\newcommand{\Intro}[1]{{#1}{\text{i}}}
\newcommand{\IntroA}[1]{{#1}{\text{i}_1}}
\newcommand{\IntroB}[1]{{#1}{\text{i}_2}}
\newcommand{\Elim}[1]{{#1}{\text{e}}}
\newcommand{\ElimA}[1]{{#1}{\text{e}_1}}
\newcommand{\ElimB}[1]{{#1}{\text{e}_2}}
\newcommand{\Set}[1]{\{ #1 \}}
\newcommand{\SET}[1]{\Bigl\{ #1 \Bigr\}}
\newcommand{\TTT}{\bm{\mathsf{T}}}
\newcommand{\FFF}{\bm{\mathsf{F}}}
\setlength{\parindent}{0pt}
% parameters: four corners, title, scribes' names
\newcommand{\handout}[6]{{
\begin{center}
\begin{minipage}{14cm}
    \setlength{\parindent}{0cm}%
    \fbox{\vbox{%
        {#1}%
        \hfill
        #2

        \center{\Large\bf{#5}}

        \emph{#3}\hfill #4
    }}%
    \vskip0pt
    \vbox{\hfill  #6}%
\end{minipage}
\\ 
\end{center}
}}

%%%%%%%%%%%%%%%%%%%%%%%%%%%%%%%%%%%%%%%%%%%%%%%%%%%%%%%%%%%%%%%%%%%%%%

\begin{document}
\handout{CS 511 Formal Methods, Fall 2024}
        {Instructor: Assaf Kfoury}
        {November 7, 2024}
        {Lucas Miguel Tassis}
        {Homework Assignment 9}

%%  THE FOLLOWING IS USED WITH THE logicproof ENVIRONMENT:
%%
\setlength{\subproofhorizspace}{.5em}
\setlength{\intersubproofvertspace}{0.5em}
\lstset{language=lean}


\section*{Exercise 1} 
First, consider that we have the signature $\{R, \approx\}$, where $R(x_1, x_2)$ means that there exists an edge between vertices $x_1$ and $x_2$. Additionally, we also consider that the graph is undirected, \emph{i.e.}, $\forall x_1 \forall x_2 (R(x, y) \to R(y, x))$.

Then, we can write an infinite set $\Gamma_{\mathrm{bipartite}}$ of first order sentences such that, for every simple graph $G$, it holds that $G \models \Gamma_{\mathrm{bipartite}}$ iff $G$ is bipartite, as:

$$\Gamma_{\mathrm{bipartite}} \defeq \{\lnot \varphi_{\{n\}} \mid n \geq 3; n\text{ is odd}\}$$

where,

$$\varphi_{\{n\}} \defeq \exists x_1 \cdots \exists x_n \left[ \left(\bigwedge_{1 \leq i, j \leq n} \lnot (x_i \approx x_j) \right) \land \left( \bigwedge_{1 \leq i \leq n-1} R(x_i, x_{i+1}) \right) \land R(x_n, x_1)\right]$$

Using the fact that $G$ is bipartite iff every cycle in $G$ has even length, we can create $\Gamma_{\mathrm{bipartite}}$ as a set of first order sentences that say that there is \textbf{no} odd cycle in $G$. To create this set, we define $\varphi_{\{n\}}$, which is true iff there is a cycle of size $n$ in the graph $G$:

\begin{enumerate}[i.]
    \item $\bigwedge\limits_{1 \leq i, j \leq n} \lnot (x_i \approx x_j)$ guarantees that every element $x_1 \cdots x_n$ are distinct elements;
    \item $\left( \bigwedge\limits_{1 \leq i \leq n-1} R(x_i, x_{i+1}) \right) \land R(x_n, x_1)$ guarantees that there is a cycle of size $n$.
\end{enumerate}

Therefore, by defining $\Gamma_{\mathrm{bipartite}}$ as the set of $\lnot \varphi_{\{n\}}$, for $n$ odd, we created an infinite set $\Gamma_{\mathrm{bipartite}}$ of first order sentences such that, for every simple graph $G$, it holds that $G \models \Gamma_{\mathrm{bipartite}}$ iff $G$ is bipartite.

\section*{Exercise 2}
Considering the three sentences:

\begin{align*}
    \varphi_1 \defeq &\ \forall x P(x, x) \\
    \varphi_2 \defeq &\ \forall x \forall y (P(x, y) \to P(y, x)) \\
    \varphi_3 \defeq &\ \forall x \forall y \forall z (P(x, y) \land P(y, z) \to P(x, z))
\end{align*}

We have to show that none of these sentences are semantically entailed by the other ones. To show that, we give an example of a $P$ that satisfies each pair of sentences but not the one left.

\begin{enumerate}[i.]
    \item $\{\varphi_1, \varphi_2 \}$: we have reflexivity and symmetry, but not transitivity.
    
    Consider the set $S \defeq \{a, b, c\}$ and the relation $P \defeq \{(a, a), (b, b), (c, c), (a, b), (b, a), (a, c), (c, a)\}$. In this example, we have reflexivity and symmetry, but we do not have transitivity. Take $(b, a)$ and $(a, c)$, we do not have $(b, c)$. 


    \item $\{\varphi_1, \varphi_3 \}$: we have reflexivity and transitivity, but not symmetry.
    
    Consider the set $S \defeq \{a, b, c\}$ and the relation $P \defeq \{(a, a), (b, b), (c, c), (a, b), (b, c), (a, c)\}$. In this example, we have reflexivity and transitivity, but we do not have symmetry. Take $(a, b)$, we do not have $(b, a)$.

    \item $\{\varphi_2, \varphi_3 \}$: we have symmetry and transitivity, but not reflexivity.
    
    Consider the set $S \defeq \{a, b, c\}$ and the relation $P \defeq \{(a, b), (b, a), (a, a), (b, b)\}$. In this example, we have symmetry and transitivity, but we do not have reflexivity. Take $c$, we do not have $(c, c)$.
\end{enumerate}

Thus, we showed that none of the sentences are semantically entailed by the other pair.

\section*{Exercise 3}
The Lean template file with the solutions is available on \href{https://github.com/lucastassis/BU-CS511/blob/main/HW09/code/HW09.lean}{GitHub}.

\section*{Exercise 4}
The Lean template file with the solutions is available on \href{https://github.com/lucastassis/BU-CS511/blob/main/HW09/code/HW09.lean}{GitHub}.

\section*{Problem 2}
The Lean template file with the solutions is available on \href{https://github.com/lucastassis/BU-CS511/blob/main/HW09/code/HW09.lean}{GitHub}.




\end{document}
