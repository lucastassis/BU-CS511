\documentclass[11pt]{article}
\usepackage{xspace,epsfig,amsmath,amsthm,amssymb,fullpage}
\topmargin   0pt
\marginparwidth 0pt
\oddsidemargin  0pt
\evensidemargin 0pt
\marginparsep 0pt
\textwidth   6.5in
\textheight  9.2in

\usepackage{tikz}
\usepackage{tikz-qtree,tikz-qtree-compat} % for parse trees

\usepackage{latexsym}
\usepackage{amssymb}
\usepackage{amsmath}
\usepackage{amsthm}
\usepackage{amsfonts}
\usepackage{bm} 
% \usepackage{prooftree}
\usepackage{flagderiv}
\usepackage{logicproof}
\usepackage{bussproofs}
\usepackage{hyperref}
\usepackage{color}
\usepackage{listings}
\usepackage{synttree} 
\usepackage{pdflscape}

\definecolor{keywordcolor}{rgb}{0.7, 0.1, 0.1}   % red
\definecolor{tacticcolor}{rgb}{0.0, 0.1, 0.6}    % blue
\definecolor{commentcolor}{rgb}{0.4, 0.4, 0.4}   % grey
\definecolor{symbolcolor}{rgb}{0.0, 0.1, 0.6}    % blue
\definecolor{sortcolor}{rgb}{0.1, 0.5, 0.1}      % green
\definecolor{attributecolor}{rgb}{0.7, 0.1, 0.1} % red

\def\lstlanguagefiles{lstlean.tex}
% set default language

\hypersetup{colorlinks=true,linkcolor=blue,urlcolor=blue}
% WARNING:
% Do NOT use package `bussproofs' and package `prooftree' at the same time,
% \begin{prooftree} ... \end{prooftree} is an environment defined in the
% package  `bussproofs', which conflicts with the name of the package
% `prooftree'.

\newcommand{\defn}{\overset{\text{\scriptsize def}}{=}}
\newcommand{\Intro}[1]{{#1}{\text{i}}}
\newcommand{\IntroA}[1]{{#1}{\text{i}_1}}
\newcommand{\IntroB}[1]{{#1}{\text{i}_2}}
\newcommand{\Elim}[1]{{#1}{\text{e}}}
\newcommand{\ElimA}[1]{{#1}{\text{e}_1}}
\newcommand{\ElimB}[1]{{#1}{\text{e}_2}}
\newcommand{\Set}[1]{\{ #1 \}}
\newcommand{\SET}[1]{\Bigl\{ #1 \Bigr\}}
\newcommand{\TTT}{\bm{\mathsf{T}}}
\newcommand{\FFF}{\bm{\mathsf{F}}}
\setlength{\parindent}{0pt}
% parameters: four corners, title, scribes' names
\newcommand{\handout}[6]{{
\begin{center}
\begin{minipage}{14cm}
    \setlength{\parindent}{0cm}%
    \fbox{\vbox{%
        {#1}%
        \hfill
        #2

        \center{\Large\bf{#5}}

        \emph{#3}\hfill #4
    }}%
    \vskip0pt
    \vbox{\hfill  #6}%
\end{minipage}
\\ 
\end{center}
}}

%%%%%%%%%%%%%%%%%%%%%%%%%%%%%%%%%%%%%%%%%%%%%%%%%%%%%%%%%%%%%%%%%%%%%%

\begin{document}
\handout{CS 511 Formal Methods, Fall 2024}
        {Instructor: Assaf Kfoury}
        {December 5, 2024}
        {Lucas Miguel Tassis}
        {Homework Assignment 12}

%%  THE FOLLOWING IS USED WITH THE logicproof ENVIRONMENT:
%%
\setlength{\subproofhorizspace}{.5em}
\setlength{\intersubproofvertspace}{0.5em}
\lstset{language=lean}


\section*{Exercise 1} 
The exercise asks us to define $X \sim Y$ by asserting the existence of a unary function $F$ from $X$ to $Y$ which is both injective and surjective. One possible way is:

\begin{align*}
    X \sim Y \defn \exists F. \Big(& \big(\forall x \in X. \forall y \in X. \forall z \in Y.(F(x) \approx z \land F(y) \approx z \to x \approx y)\big) \land \\
    & \big(\forall y \in Y. \exists x \in X. (F(x) \approx y)\big)\Big)
\end{align*}

The first line says that $F$ is injective from $X$ to $Y$, and the second says that $F$ is surjective from $X$ to $Y$.


\section*{Exercise 2}

\begin{enumerate}
    \item The first item asks us to define a 2nd-order sentence $\Psi_{\text{countable-infty}}(Y)$ s.t. $\mathcal{A} \models \Psi_{\text{countable-infty}}$ iff $\mathcal{A}$ is countably infinite. Similarly to the sentence in the slide, we can define as:
    $$\Psi_{\text{countable-infty}}(Y) \defn \Psi_{\text{infty}}(Y) \land \big(\forall X \subseteq Y.\ \Psi_{\text{infty}}(X) \to (X \sim Y) \big),$$
    where
    \begin{align*}
        \Psi_{\text{infty}}(X) \defn \exists F. \Big(& \big(\forall x \in X. \forall y \in X. \forall z \in X.(F(x) \approx z \land F(y) \approx z \to x \approx y)\big) \land \\
        & \big(\exists y \in X. \forall x \in X. \lnot (F(x) \approx y)\big)\Big),
    \end{align*}
    and $X \sim Y$ is as defined in Exercise 1.
    In this case, the model should be infinite, and also there is a bijection from every subset $X$ of $Y$ from $X$ to $Y$.
    \item The second item asks us to define a 2nd-order sentence $\Psi_{\text{uncountable}}(Y)$ s.t. $\mathcal{A} \models \Psi_{\text{uncountable}}$ iff $\mathcal{A}$ is uncountably infinite. We can define the sentence as:
    $$\Psi_{\text{uncountable}}(Y) \defn \Psi_{\text{infty}}(Y) \land \lnot \big(\forall X \subseteq Y.\ \Psi_{\text{infty}}(X) \to (X \sim Y) \big),$$
    where $\Psi_{\text{infty}}(Y)$ and $X \sim Y$ are the same as defined in Item 1.
    We can also rewrite the sentence by pushing the negation:
    $$\Psi_{\text{uncountable}}(Y) \defn \Psi_{\text{infty}}(Y) \land \big(\exists X \subseteq Y.\ \Psi_{\text{infty}}(X) \land \lnot (X \sim Y) \big),$$
    In this case, the model should still be infinite, however, there is not a bijection from every subset $X$ of $Y$ from $X$ to $Y$.
\end{enumerate}

\section*{Exercise 3}

The exercise asks us to define a second-order wff $\theta (x, y)$, such that $\theta(x,y)$ iff ``no binary predicate $Y$ can discern $x$ and $y$''. One possible way is:

$$\theta(x, y) \defn \forall Y.\Big(\forall c. \big(Y(x, c) \leftrightarrow Y(y, c) \land Y(c, x) \leftrightarrow Y(c, y)\big)\Big)$$

The idea behind the wff is that $x$ and $y$ are identical to $\theta$ iff $x$ and $y$ satisfy the same binary predicates, and each iff in the wff guarantees this.

\section*{Exercise 4}
The Lean template file with the solutions is available on \href{https://github.com/lucastassis/BU-CS511/blob/main/HW12/code/HW12.lean}{GitHub}.

\section*{Exercise 5}
The Lean template file with the solutions is available on \href{https://github.com/lucastassis/BU-CS511/blob/main/HW12/code/HW12.lean}{GitHub}.

\section*{Exercise 6}
The Lean template file with the solutions is available on \href{https://github.com/lucastassis/BU-CS511/blob/main/HW12/code/HW12.lean}{GitHub}.

\section*{Problem 2}
The Lean template file with the solutions is available on \href{https://github.com/lucastassis/BU-CS511/blob/main/HW12/code/HW12.lean}{GitHub}.




\end{document}
