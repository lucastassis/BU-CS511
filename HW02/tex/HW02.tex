\documentclass[11pt]{article}
\usepackage{xspace,epsfig,amsmath,amsthm,amssymb,fullpage}
\topmargin   0pt
\marginparwidth 0pt
\oddsidemargin  0pt
\evensidemargin 0pt
\marginparsep 0pt
\textwidth   6.5in
\textheight  9.2in

\usepackage{tikz}
\usepackage{tikz-qtree,tikz-qtree-compat} % for parse trees

\usepackage{latexsym}
\usepackage{amssymb}
\usepackage{amsmath}
\usepackage{amsthm}
\usepackage{amsfonts}
\usepackage{bm} 
% \usepackage{prooftree}
\usepackage{flagderiv}
\usepackage{logicproof}
\usepackage{bussproofs}
\usepackage{hyperref}
\usepackage{color}
\usepackage{listings}
\definecolor{keywordcolor}{rgb}{0.7, 0.1, 0.1}   % red
\definecolor{tacticcolor}{rgb}{0.0, 0.1, 0.6}    % blue
\definecolor{commentcolor}{rgb}{0.4, 0.4, 0.4}   % grey
\definecolor{symbolcolor}{rgb}{0.0, 0.1, 0.6}    % blue
\definecolor{sortcolor}{rgb}{0.1, 0.5, 0.1}      % green
\definecolor{attributecolor}{rgb}{0.7, 0.1, 0.1} % red

\def\lstlanguagefiles{lstlean.tex}
% set default language

\hypersetup{colorlinks=true,linkcolor=blue,urlcolor=blue}
% WARNING:
% Do NOT use package `bussproofs' and package `prooftree' at the same time,
% \begin{prooftree} ... \end{prooftree} is an environment defined in the
% package  `bussproofs', which conflicts with the name of the package
% `prooftree'.

\newcommand{\defn}{\overset{\text{\scriptsize def}}{=}}
\newcommand{\Intro}[1]{{#1}{\text{i}}}
\newcommand{\IntroA}[1]{{#1}{\text{i}_1}}
\newcommand{\IntroB}[1]{{#1}{\text{i}_2}}
\newcommand{\Elim}[1]{{#1}{\text{e}}}
\newcommand{\ElimA}[1]{{#1}{\text{e}_1}}
\newcommand{\ElimB}[1]{{#1}{\text{e}_2}}
\newcommand{\Set}[1]{\{ #1 \}}
\newcommand{\SET}[1]{\Bigl\{ #1 \Bigr\}}
\newcommand{\TTT}{\bm{\mathsf{T}}}
\newcommand{\FFF}{\bm{\mathsf{F}}}
\setlength{\parindent}{0pt}
% parameters: four corners, title, scribes' names
\newcommand{\handout}[6]{{
\begin{center}
\begin{minipage}{14cm}
    \setlength{\parindent}{0cm}%
    \fbox{\vbox{%
        {#1}%
        \hfill
        #2

        \center{\Large\bf{#5}}

        \emph{#3}\hfill #4
    }}%
    \vskip0pt
    \vbox{\hfill  #6}%
\end{minipage}
\\ 
\end{center}
}}

%%%%%%%%%%%%%%%%%%%%%%%%%%%%%%%%%%%%%%%%%%%%%%%%%%%%%%%%%%%%%%%%%%%%%%

\begin{document}
\handout{CS 511 Formal Methods, Fall 2024}
        {Instructor: Assaf Kfoury}
        {September 19, 2024}
        {Lucas Miguel Tassis}
        {Homework Assignment 2}

%%  THE FOLLOWING IS USED WITH THE logicproof ENVIRONMENT:
%%
\setlength{\subproofhorizspace}{.5em}
\setlength{\intersubproofvertspace}{0.5em}
\lstset{language=lean}


\section*{Exercise 1}
(Some ideas were from the example of the proof of length of two lists on \href{https://en.wikipedia.org/wiki/Structural_induction}{Wikipedia})

The exercise asks us to use structural induction on $t \in A^*$ to prove the property $P(t)$ defined by:

$$P(t) \stackrel{\text{def}}{=} \text{for all }s\in A^* \text{ it holds that reverse}(s \cdot t) = \text{reverse}(t) \cdot \text{reverse}(s)$$

To prove this property by structural induction, we also have to use the definition of the reverse function. Using the definition on page 7 in Lecture Slides 6, we have that reverse$(s \cdot x) \stackrel{\text{def}}{=}x \cdot \text{reverse}(s)$ and reverse$(\varepsilon) \stackrel{\text{def}}{=}\varepsilon$. 

\paragraph{Base step:} In the base case, we have that t = $\varepsilon$ (empty string), thus:

\begin{align*}
    \text{reverse}(s \cdot t) &= \text{reverse}(s \cdot \varepsilon) \\
    &= \text{reverse}(s) \\
    &= \varepsilon \cdot \text{reverse}(s) \\
    &= \text{reverse}(t) \cdot \text{reverse}(s)
\end{align*}

Thus, we have that the property holds up in the base case.

\paragraph*{Inductive step:} Consider that we denote the string $t = w \cdot x$, where $x$ is the last character of the string $t$. To prove by structural induction, we will assume that $P(w)$ is true, and prove that the property still holds for $P(t=w \cdot x)$. If we apply $\text{reverse}(s \cdot t)$:

\begin{align*}
    \text{reverse}(s \cdot t) &= \text{reverse}(s \cdot (w \cdot x)) \\
    &= \text{reverse}((s \cdot w) \cdot x) \\
    &= x \cdot \text{reverse}(s \cdot w) && \text{reverse definition}\\
    &= x \cdot (\text{reverse}(w) \cdot \text{reverse}(s)) && \text{induction hypothesis}\\
    &= (x \cdot \text{reverse}(w)) \cdot \text{reverse}(s) \\
    &= \text{reverse} (w \cdot x) \cdot \text{reverse}(s) \\
    &= \text{reverse}(t) \cdot \text{reverse}(s)
\end{align*}

Since the property still holds for $P(t)$, we proved by structural induction that:

$$P(t) \stackrel{\text{def}}{=} \text{for all }s\in A^* \text{ it holds that reverse}(s \cdot t) = \text{reverse}(t) \cdot \text{reverse}(s)$$

\section*{Exercise 2}
(From LCS, page 87: Exercise 1.4.15)

We have to use mathematical induction on $n$ to show that:

$$((\phi_1 \land ( \phi_2 \land (\cdots \land \phi_n) \cdots) \to \psi) \to ((\phi_1 \to (\phi_2 \to (\cdots (\phi_n \to \psi) \cdots))))$$

\paragraph*{Base step:} First, we have to show that the statement holds for the base case, which is $n = 1$. In this case, we have $(\phi_1 \to \psi) \to (\phi_1 \to \psi)$, which is true. So the statement holds for the base case.

\paragraph*{Inductive step:} Now, we have to show that if the statement is holds for $n=k$ (induction hypothesis), then $k+1$ also holds. Using natural deduction on $k + 1$:

\begin{logicproof}{3}
    ((\phi_1\land(\phi_2\land(\cdots\land(\phi_k\land\phi_{k+1})\cdots)) \to \psi & premise \\
    \begin{subproof}
        (\phi_1\land(\phi_2 \land(\cdots \land \phi_k)\cdots) & assumption\\
        \begin{subproof}
            \phi_{k+1} & assumption \\
            (\phi_1\land(\phi_2\land(\cdots\land(\phi_k\land\phi_{k+1})\cdots) & $\Intro{\land}$ 2,3 \\
            \psi & $\Elim{\to}$ 1,4
        \end{subproof}
        \phi_{k+1} \to \psi & $\Intro{\to}$ 3--5
    \end{subproof}
    (\phi_1\land(\phi_2 \land(\cdots \land \phi_k)\cdots) \to (\phi_{k+1} \to \psi) & $\Intro{\to}$ 2--6 \\
    (\phi_1 \to (\phi_2 \to (\cdots (\phi_k \to (\phi_{k+1} \to \psi) \cdots))) & induction hypothesis
\end{logicproof}

Thus, we showed by mathematical induction that the statement holds.

\section*{Problem 1}

\subsection*{(b) (LEM) is derivable from (PBC)}

\begin{logicproof}{3}
    \begin{subproof}
        \lnot (p \lor \lnot p) & assumption \\
        \begin{subproof}
            p & assumption \\
            p \lor \lnot p & $\IntroA{\lor}$ 2\\
            \bot & $\Elim{\lnot}$ 1, 3
        \end{subproof}
        \lnot p & $\Intro{\lnot}$ 2--4 \\
        p \lor \lnot p & $\IntroB{\lor}$ 5 \\
        \bot & $\Elim{\lnot}$ 1, 6
    \end{subproof}
    p \lor \lnot p & PBC 1--7
\end{logicproof}

\newpage
\subsection*{(c) ($\lnot\lnot$E) is derivable from (LEM)}

The only way I found of deriving ($\lnot\lnot$E) from (LEM) was by also using \textit{disjunctive syllogism} (\href{https://en.wikipedia.org/wiki/Disjunctive_syllogism}{Wikipedia}). Disjunctive syllogism is: $p \lor q, \lnot p \vdash q$. I tried proving it (in order to use it in the exercise), but couldn't do it without using ($\lnot\lnot$E). I don't know if there is another way of proving disjunctive syllogism without either ($\lnot\lnot$E) or (PBC). Most of the places I searched also used one of the two to prove it (\href{https://math.stackexchange.com/questions/3337855/proof-disjunctive-syllogism-using-natural-deduction}{for example}). I decided to put the proof here because it was the only one I could find, but I think it might involve the use of other rules beside LEM (which probably goes against the idea of the exercise).

\begin{logicproof}{3}
    \lnot p \lor p & LEM \\
    \begin{subproof}
        \lnot \lnot p & assumption \\
        p & disjunctive syllogism
    \end{subproof}
    \lnot \lnot p \to p & $\Intro{\to}$ 2--3
\end{logicproof}

\section*{Exercise 3}
The Lean template file with the solutions is available on \href{https://github.com/lucastassis/BU-CS511/blob/main/HW02/code/HW02.lean}{GitHub}.

\section*{Exercise 4}
The Lean template file with the solutions is available on \href{https://github.com/lucastassis/BU-CS511/blob/main/HW02/code/HW02.lean}{GitHub}.

\section*{Problem 2}
The Lean template file with the solutions is available on \href{https://github.com/lucastassis/BU-CS511/blob/main/HW02/code/HW02.lean}{GitHub}.

\end{document}
