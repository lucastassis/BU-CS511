\documentclass[11pt]{article}
\usepackage{xspace,epsfig,amsmath,amsthm,amssymb,fullpage}
\topmargin   0pt
\marginparwidth 0pt
\oddsidemargin  0pt
\evensidemargin 0pt
\marginparsep 0pt
\textwidth   6.5in
\textheight  9.2in

\usepackage{tikz}
\usepackage{tikz-qtree,tikz-qtree-compat} % for parse trees

\usepackage{latexsym}
\usepackage{amssymb}
\usepackage{amsmath}
\usepackage{amsthm}
\usepackage{amsfonts}
\usepackage{bm} 
% \usepackage{prooftree}
\usepackage{flagderiv}
\usepackage{logicproof}
\usepackage{bussproofs}
\usepackage{hyperref}
\usepackage{color}
\usepackage{listings}
\usepackage{synttree} 

\definecolor{keywordcolor}{rgb}{0.7, 0.1, 0.1}   % red
\definecolor{tacticcolor}{rgb}{0.0, 0.1, 0.6}    % blue
\definecolor{commentcolor}{rgb}{0.4, 0.4, 0.4}   % grey
\definecolor{symbolcolor}{rgb}{0.0, 0.1, 0.6}    % blue
\definecolor{sortcolor}{rgb}{0.1, 0.5, 0.1}      % green
\definecolor{attributecolor}{rgb}{0.7, 0.1, 0.1} % red

\def\lstlanguagefiles{lstlean.tex}
% set default language

\hypersetup{colorlinks=true,linkcolor=blue,urlcolor=blue}
% WARNING:
% Do NOT use package `bussproofs' and package `prooftree' at the same time,
% \begin{prooftree} ... \end{prooftree} is an environment defined in the
% package  `bussproofs', which conflicts with the name of the package
% `prooftree'.

\newcommand{\defn}{\overset{\text{\scriptsize def}}{=}}
\newcommand{\Intro}[1]{{#1}{\text{i}}}
\newcommand{\IntroA}[1]{{#1}{\text{i}_1}}
\newcommand{\IntroB}[1]{{#1}{\text{i}_2}}
\newcommand{\Elim}[1]{{#1}{\text{e}}}
\newcommand{\ElimA}[1]{{#1}{\text{e}_1}}
\newcommand{\ElimB}[1]{{#1}{\text{e}_2}}
\newcommand{\Set}[1]{\{ #1 \}}
\newcommand{\SET}[1]{\Bigl\{ #1 \Bigr\}}
\newcommand{\TTT}{\bm{\mathsf{T}}}
\newcommand{\FFF}{\bm{\mathsf{F}}}
\setlength{\parindent}{0pt}
% parameters: four corners, title, scribes' names
\newcommand{\handout}[6]{{
\begin{center}
\begin{minipage}{14cm}
    \setlength{\parindent}{0cm}%
    \fbox{\vbox{%
        {#1}%
        \hfill
        #2

        \center{\Large\bf{#5}}

        \emph{#3}\hfill #4
    }}%
    \vskip0pt
    \vbox{\hfill  #6}%
\end{minipage}
\\ 
\end{center}
}}

%%%%%%%%%%%%%%%%%%%%%%%%%%%%%%%%%%%%%%%%%%%%%%%%%%%%%%%%%%%%%%%%%%%%%%

\begin{document}
\handout{CS 511 Formal Methods, Fall 2024}
        {Instructor: Assaf Kfoury}
        {October 17, 2024}
        {Lucas Miguel Tassis}
        {Homework Assignment 6}

%%  THE FOLLOWING IS USED WITH THE logicproof ENVIRONMENT:
%%
\setlength{\subproofhorizspace}{.5em}
\setlength{\intersubproofvertspace}{0.5em}
\lstset{language=lean}


\section*{Exercise 1} 

We have to show that every finite subset $X \subseteq \mathbb{N}$ is first-order definable in the structure $(\mathbb{N}; <)$. Consider $X = \{x_1, \cdots, x_k\}$. We want to define:

\begin{equation*}
    \varphi_X(x) \stackrel{\text{def}}{=} 
    \begin{cases}
        \text{true,} & \text{if } x \in X \\
        \text{false,} & \text{otherwise}
    \end{cases}
\end{equation*}

Thus, 

$$\varphi_X(x) \stackrel{\text{def}}{=}  \bigvee_{i=1}^{k} \psi_{\{x_k\}}(x)$$
where,


$$\psi_{\{x_k\}}(x) \stackrel{\text{def}}{=} \exists {y_0} \cdots \exists {y_{x_k - 1}} \left[ \varphi_{\{0\}}(y_0) \land \varphi_{\scriptscriptstyle\text{SUCC}}(y_0, y_1) \land \cdots \land \varphi_{\scriptscriptstyle\text{SUCC}}(y_{x_k - 1}, x)\right]$$

and $\varphi_{\{0\}}(x)$, $\varphi_{\scriptscriptstyle\text{SUCC}}(x, y)$ are the wffs for the constant 0 and the successor function, respectively (proved in Exercises 1 and 2 of Lecture Slides 22 -- Appendix). 

\vspace{0.25cm}

Notice that $\psi_{\{x_k\}}$ can write a wff for any natural number $x_k$. Therefore, $\varphi_X(x)$ will be true for any $x \in X$, and false otherwise (because any $x \notin X$ will not be a part of $\varphi_X(x)$).

\vspace{0.25cm}
PS: When $x_k = 0$ we would like to have $\psi_{\{0\}} (x) = \varphi_{\{0\}} (x)$, but I did not know if I could break the expression in cases when $x_k = 0$ and when $x_k \neq 0$. If we can, I think a more formal definition (or at least more ``organized'') of $\psi_{\{x_k\}}(x)$ would be:

\begin{equation*}
    \psi_{\{x_k\}}(x) \stackrel{\text{def}}{=} 
    \begin{cases}
        \varphi_{\{0\}}(x), & \text{if } x_k = 0 \\
        \exists {y_1} \cdots \exists {y_{x_k}} \left[ \varphi_{\{0\}}(y_1) \land \varphi_{\scriptscriptstyle\text{SUCC}}(y_1, y_2) \land \cdots \land \varphi_{\scriptscriptstyle\text{SUCC}}(y_{x_k}, x)\right], & \text{otherwise}
    \end{cases}
\end{equation*}



% \begin{equation*}
%     \psi(x) = 
%     \begin{cases}
%         \text{true,} & $if $ x \in X \\
%         \text{false,} & \text{otherwise}
%     \end{cases}
% \end{equation*}

\section*{Exercise 2}

We have to show that the predicate $\texttt{prime}: \mathbb{N} \to \{\text{true}, \text{false}\}$ is first order definable in the structure $(\mathbb{N}; |, +, 0)$. Thus, we need to define:

\begin{equation*}
    \texttt{prime}(n) \stackrel{\text{def}}{=} 
    \begin{cases}
        \text{true,} & \text{if } n \text{ is a prime number} \\
        \text{false,} & \text{otherwise}
    \end{cases}
\end{equation*}

For that, let us define equality first. We define $\texttt{equal}(m, n)$, which is true if $m = n$, and false otherwise as:

$$\texttt{equal}(m, n) \stackrel{\text{def}}{=} (m | n) \land (n | m)$$

Additionally, we can define $1 \stackrel{\text{def}}{=} \texttt{succ}(0)$ (proved in Exercise 2 of Lecture Slides 22 -- Appendix). Notice that we can prove $<$ using $(\mathbb{N}; +, 0)$, as shown in Exercise 4 of  Lecture Slides 22 -- Appendix (so we can use the successor result). Then, we can define \texttt{prime}(n) as:

$$\texttt{prime}(n)\stackrel{\text{def}}{=} \lnot \texttt{equal}(n, 0)\land \lnot \texttt{equal}(n, 1) \land \forall y (y | n \to \texttt{equal}(y, 1) \lor \texttt{equal}(y, n))$$


\section*{Exercise 3}
The Lean template file with the solutions is available on \href{https://github.com/lucastassis/BU-CS511/blob/main/HW06/code/HW06.lean}{GitHub}.

\section*{Exercise 4}
The Lean template file with the solutions is available on \href{https://github.com/lucastassis/BU-CS511/blob/main/HW06/code/HW06.lean}{GitHub}.

\section*{Problem 2}
The Lean template file with the solutions is available on \href{https://github.com/lucastassis/BU-CS511/blob/main/HW06/code/HW06.lean}{GitHub}.






\end{document}
