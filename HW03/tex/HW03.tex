\documentclass[11pt]{article}
\usepackage{xspace,epsfig,amsmath,amsthm,amssymb,fullpage}
\topmargin   0pt
\marginparwidth 0pt
\oddsidemargin  0pt
\evensidemargin 0pt
\marginparsep 0pt
\textwidth   6.5in
\textheight  9.2in

\usepackage{tikz}
\usepackage{tikz-qtree,tikz-qtree-compat} % for parse trees

\usepackage{latexsym}
\usepackage{amssymb}
\usepackage{amsmath}
\usepackage{amsthm}
\usepackage{amsfonts}
\usepackage{bm} 
% \usepackage{prooftree}
\usepackage{flagderiv}
\usepackage{logicproof}
\usepackage{bussproofs}
\usepackage{hyperref}
\usepackage{color}
\usepackage{listings}
\usepackage{synttree} 

\definecolor{keywordcolor}{rgb}{0.7, 0.1, 0.1}   % red
\definecolor{tacticcolor}{rgb}{0.0, 0.1, 0.6}    % blue
\definecolor{commentcolor}{rgb}{0.4, 0.4, 0.4}   % grey
\definecolor{symbolcolor}{rgb}{0.0, 0.1, 0.6}    % blue
\definecolor{sortcolor}{rgb}{0.1, 0.5, 0.1}      % green
\definecolor{attributecolor}{rgb}{0.7, 0.1, 0.1} % red

\def\lstlanguagefiles{lstlean.tex}
% set default language

\hypersetup{colorlinks=true,linkcolor=blue,urlcolor=blue}
% WARNING:
% Do NOT use package `bussproofs' and package `prooftree' at the same time,
% \begin{prooftree} ... \end{prooftree} is an environment defined in the
% package  `bussproofs', which conflicts with the name of the package
% `prooftree'.

\newcommand{\defn}{\overset{\text{\scriptsize def}}{=}}
\newcommand{\Intro}[1]{{#1}{\text{i}}}
\newcommand{\IntroA}[1]{{#1}{\text{i}_1}}
\newcommand{\IntroB}[1]{{#1}{\text{i}_2}}
\newcommand{\Elim}[1]{{#1}{\text{e}}}
\newcommand{\ElimA}[1]{{#1}{\text{e}_1}}
\newcommand{\ElimB}[1]{{#1}{\text{e}_2}}
\newcommand{\Set}[1]{\{ #1 \}}
\newcommand{\SET}[1]{\Bigl\{ #1 \Bigr\}}
\newcommand{\TTT}{\bm{\mathsf{T}}}
\newcommand{\FFF}{\bm{\mathsf{F}}}
\setlength{\parindent}{0pt}
% parameters: four corners, title, scribes' names
\newcommand{\handout}[6]{{
\begin{center}
\begin{minipage}{14cm}
    \setlength{\parindent}{0cm}%
    \fbox{\vbox{%
        {#1}%
        \hfill
        #2

        \center{\Large\bf{#5}}

        \emph{#3}\hfill #4
    }}%
    \vskip0pt
    \vbox{\hfill  #6}%
\end{minipage}
\\ 
\end{center}
}}

%%%%%%%%%%%%%%%%%%%%%%%%%%%%%%%%%%%%%%%%%%%%%%%%%%%%%%%%%%%%%%%%%%%%%%

\begin{document}
\handout{CS 511 Formal Methods, Fall 2024}
        {Instructor: Assaf Kfoury}
        {September 26, 2024}
        {Lucas Miguel Tassis}
        {Homework Assignment 3}

%%  THE FOLLOWING IS USED WITH THE logicproof ENVIRONMENT:
%%
\setlength{\subproofhorizspace}{.5em}
\setlength{\intersubproofvertspace}{0.5em}
\lstset{language=lean}


\section*{Exercise 1} 

$\phi_1 \stackrel{\text{def}}{=} \lnot ((p \land q) \land r) \to ((\lnot p \lor \lnot q) \lor \lnot r)$ 

\begin{logicproof}{3}
    \begin{subproof}
        \lnot ((p \land q) \land r) & assumption \\
        \begin{subproof}
            \lnot ((\lnot p \lor \lnot q) \lor \lnot r) & assumption \\
            \begin{subproof}
                \lnot p & assumption \\
                \lnot p \lor \lnot q & $\Intro{\lor}$ 3 \\
                (\lnot p \lor \lnot q) \lor \lnot r & $\Intro{\lor}$ 4\\
                \bot & $\Elim{\lnot}$ 2, 5
            \end{subproof}
            \lnot \lnot p & $\Intro{\lnot}$ 3--6 \\
            \begin{subproof}
                \lnot q & assumption \\
                \lnot p \lor \lnot q & $\Intro{\lor}$ 8 \\
                (\lnot p \lor \lnot q) \lor \lnot r & $\Intro{\lor}$ 9\\
                \bot & $\Elim{\lnot}$ 2, 10
            \end{subproof}
            \lnot \lnot q & $\Intro{\lnot}$ 8--11 \\
            \begin{subproof}
                \lnot r & assumption \\
                \lnot q \lor \lnot r & $\Intro{\lor}$ 13 \\
                (\lnot p \lor \lnot q) \lor \lnot r & $\Intro{\lor}$ 14\\
                \bot & $\Elim{\lnot}$ 2, 15
            \end{subproof}
            \lnot \lnot r & $\Intro{\lnot}$ 13--16 \\
            p & $\Elim{\lnot\lnot}$ 7 \\
            q & $\Elim{\lnot\lnot}$ 12 \\
            r & $\Elim{\lnot\lnot}$ 17 \\
            p \land q & $\Intro{\land}$ 18, 19 \\
            (p \land q) \land r & $\Intro{\land}$ 20, 21 \\
            \bot & $\Elim{\lnot}$ 1, 22
        \end{subproof}
        \lnot \lnot ((\lnot p \lor \lnot q) \lor \lnot r) & $\Intro{\lnot}$ 2--23 \\
        ((\lnot p \lor \lnot q) \lor \lnot r) & $\Elim{\lnot\lnot}$ 24
    \end{subproof}
    \lnot ((p \land q) \land r) \to ((\lnot p \lor \lnot q) \lor \lnot r) & $\Intro{\to}$ 1 --25
\end{logicproof}

\section*{Exercise 2}
To show that the de Morgan's law for three variables $\phi_1 \stackrel{\text{def}}{=} \lnot ((p \land q) \land r) \to ((\lnot p \lor \lnot q) \lor \lnot r)$ is valid, we need to show that its negation is a contradiction, thus:

\begin{center}
    \synttree{8 \branchheight{.45in} \childsidesep{5em} \childattachsep{1em}}
    [ $\lnot (\lnot ((p \land q) \land r) \to ((\lnot p \lor \lnot q) \lor \lnot r))$ 
        [ $\lnot ((p \land q) \land r)$
        [ $\lnot ((\lnot p \lor \lnot q) \lor \lnot r)$
        [ $\lnot (\lnot p \lor \lnot q)$
        [ $\lnot \lnot r$
        [ $\lnot \lnot p$ 
        [ $\lnot \lnot q$ 
        [ $r$
        [ $p$
        [ $q$
        [ $\lnot (p \land q)$
            [$\lnot p$
            [\bf{X}]]
            [$\lnot q$
            [\bf{X}]]
        ]
        [ $\lnot r$
            [\bf{X}]
        ]
        ]]]]]]]]]
    ]
    \end{center}

Since all paths are closed, the negation of the de Morgan's law is a contradiction. Thus, we proved that the de Morgan's law for 3 variables is a tautology.


\section*{Exercise 3}
The Lean template file with the solutions is available on \href{https://github.com/lucastassis/BU-CS511/blob/main/HW03/code/HW03.lean}{GitHub}.

\section*{Exercise 4}
The Lean template file with the solutions is available on \href{https://github.com/lucastassis/BU-CS511/blob/main/HW03/code/HW03.lean}{GitHub}.


\end{document}
