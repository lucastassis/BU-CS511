\documentclass[11pt]{article}
\usepackage{xspace,epsfig,amsmath,amsthm,amssymb,fullpage}
\topmargin   0pt
\marginparwidth 0pt
\oddsidemargin  0pt
\evensidemargin 0pt
\marginparsep 0pt
\textwidth   6.5in
\textheight  9.2in

\usepackage{tikz}
\usepackage{tikz-qtree,tikz-qtree-compat} % for parse trees

\usepackage{latexsym}
\usepackage{amssymb}
\usepackage{amsmath}
\usepackage{amsthm}
\usepackage{amsfonts}
\usepackage{bm} 
% \usepackage{prooftree}
\usepackage{flagderiv}
\usepackage{logicproof}
\usepackage{bussproofs}
\usepackage{hyperref}
\usepackage{color}
\usepackage{listings}
\definecolor{keywordcolor}{rgb}{0.7, 0.1, 0.1}   % red
\definecolor{tacticcolor}{rgb}{0.0, 0.1, 0.6}    % blue
\definecolor{commentcolor}{rgb}{0.4, 0.4, 0.4}   % grey
\definecolor{symbolcolor}{rgb}{0.0, 0.1, 0.6}    % blue
\definecolor{sortcolor}{rgb}{0.1, 0.5, 0.1}      % green
\definecolor{attributecolor}{rgb}{0.7, 0.1, 0.1} % red

\def\lstlanguagefiles{lstlean.tex}
% set default language

\hypersetup{colorlinks=true,linkcolor=blue,urlcolor=blue}
% WARNING:
% Do NOT use package `bussproofs' and package `prooftree' at the same time,
% \begin{prooftree} ... \end{prooftree} is an environment defined in the
% package  `bussproofs', which conflicts with the name of the package
% `prooftree'.

\newcommand{\defn}{\overset{\text{\scriptsize def}}{=}}
\newcommand{\Intro}[1]{{#1}{\text{i}}}
\newcommand{\IntroA}[1]{{#1}{\text{i}_1}}
\newcommand{\IntroB}[1]{{#1}{\text{i}_2}}
\newcommand{\Elim}[1]{{#1}{\text{e}}}
\newcommand{\ElimA}[1]{{#1}{\text{e}_1}}
\newcommand{\ElimB}[1]{{#1}{\text{e}_2}}
\newcommand{\Set}[1]{\{ #1 \}}
\newcommand{\SET}[1]{\Bigl\{ #1 \Bigr\}}
\newcommand{\TTT}{\bm{\mathsf{T}}}
\newcommand{\FFF}{\bm{\mathsf{F}}}

% parameters: four corners, title, scribes' names
\newcommand{\handout}[6]{{
\begin{center}
\begin{minipage}{14cm}
    \setlength{\parindent}{0cm}%
    \fbox{\vbox{%
        {#1}%
        \hfill
        #2

        \center{\Large\bf{#5}}

        \emph{#3}\hfill #4
    }}%
    \vskip0pt
    \vbox{\hfill  #6}%
\end{minipage}
\\ 
\end{center}
}}

%%%%%%%%%%%%%%%%%%%%%%%%%%%%%%%%%%%%%%%%%%%%%%%%%%%%%%%%%%%%%%%%%%%%%%

\begin{document}
\handout{CS 511 Formal Methods, Fall 2024}
        {Instructor: Assaf Kfoury}
        {September 12, 2024}
        {Lucas Miguel Tassis}
        {Homework Assignment 01}

%%  THE FOLLOWING IS USED WITH THE logicproof ENVIRONMENT:
%%
\setlength{\subproofhorizspace}{.5em}
\setlength{\intersubproofvertspace}{0.5em}
\lstset{language=lean}


\section*{Exercise 1}

\subsection*{Exercise 1.2.1 -- (h): $p \vdash (p \to q) \to q $}
\begin{logicproof}{3}
    p & premise \\
    \begin{subproof}
        p \to q & assumption \\
        q & $\Elim{\to}$ \ 1, 2
    \end{subproof}
    (p \to q) \to q & $\Intro{\to}$ \ 2--3
\end{logicproof}

\subsection*{Exercise 1.2.1 -- (i): $(p \to r) \land  (q \to r) \vdash p \land q \to r$}
\begin{logicproof}{3}
    (p \to r) \land (q \to r) & premise \\
    p \to r & $\ElimA{\land}$ 1 \\
    \begin{subproof}
        p \land q & assumption \\
        p & $\ElimA{\land}$ 3 \\
        r & $\Elim{\to}$ 2, 4
    \end{subproof}
    (p \land q) \to r & $\Intro{\to}$ 3--5
\end{logicproof}

\subsection*{Exercise 1.2.1 -- (j): $q \to r \vdash (p \to q) \to (p \to r)$}
\begin{logicproof}{3}
    q \to r & premise \\
    \begin{subproof}
        p \to q & assumption \\
        \begin{subproof}
            p & assumption \\
            q & $\Elim{\to}$ 2, 3 \\
            r & $\Elim{\to}$ 1, 4
        \end{subproof}
        p \to r & $\Intro{\to}$ 3--5
    \end{subproof}
    (p \to q) \to (p \to r) & $\Intro{\to}$ 2--6
\end{logicproof}

\section*{Exercise 2}

\subsection*{Exercise 1.4.2 -- (g): $((p \to q) \to p) \to p$}
\[
\begin{array}{ c | c || c | c | c}
 p & q & p \to q & (p \to q) \to p &  ((p \to q) \to p) \to p
\\ \hline 
 \TTT & \TTT & \TTT & \TTT & \TTT\\ \hline  
 \TTT & \FFF & \FFF & \TTT & \TTT\\ \hline  
 \FFF & \TTT & \TTT & \FFF & \TTT\\ \hline  
 \FFF & \FFF & \TTT & \FFF & \TTT  
\end{array}
\]

\subsection*{Exercise 1.4.2 -- (h): $((p \lor q) \to r) \to ((p \to r) \lor (q \to r))$}
\[
\begin{array}{ c | c | c || c | c | c | c | c | c}
    p & q & r & p \lor q & (p \lor q) \to r& p \to r & q \to r & (p \to r) \lor (q \to r) & ((p \lor q) \to r) \to ((p \to r) \lor (q \to r))\\ \hline 
    \TTT & \TTT & \TTT & \TTT & \TTT & \TTT & \TTT & \TTT & \TTT\\ \hline  
    \TTT & \TTT & \FFF & \TTT & \FFF & \FFF & \FFF & \FFF & \TTT\\ \hline  
    \TTT & \FFF & \FFF & \TTT & \FFF & \FFF & \TTT & \TTT & \TTT\\ \hline  
    \TTT & \FFF & \TTT & \TTT & \TTT & \TTT & \TTT & \TTT & \TTT\\ \hline  
    \FFF & \TTT & \TTT & \TTT & \TTT & \TTT & \TTT & \TTT & \TTT\\ \hline  
    \FFF & \FFF & \TTT & \FFF & \TTT & \TTT & \TTT & \TTT & \TTT\\ \hline  
    \FFF & \TTT & \FFF & \TTT & \FFF & \TTT & \FFF & \TTT & \TTT\\ \hline
    \FFF & \FFF & \FFF & \FFF & \TTT & \TTT & \TTT & \TTT & \TTT
   \end{array}
\]

\subsection*{Exercise 1.4.2 -- (i): $(p \to q) \to (\neg{p} \to \neg{q})$}
\[
\begin{array}{ c | c || c | c | c | c | c}
 p & q & \neg{p} & \neg{q}& p \to q & \neg{p} \to \neg{q} & (p \to q) \to (\neg{p} \to \neg{q})
\\ \hline 
 \TTT & \TTT & \FFF & \FFF & \TTT & \TTT & \TTT \\ \hline  
 \TTT & \FFF & \FFF & \TTT & \FFF & \TTT & \TTT \\ \hline  
 \FFF & \TTT & \TTT & \FFF & \TTT & \FFF & \FFF \\ \hline  
 \FFF & \FFF & \TTT & \TTT & \TTT & \TTT & \TTT
\end{array}
\]

\section*{Exercise 3}
Link to code on \href{https://github.com/lucastassis/BU-CS511/HW01/code/HW01.lean}{GitHub}. The solution was:

\begin{lstlisting}
-- Example 1.3.4
example {w : ℚ} (h1 : 3 * w + 1 = 4) : w = 1 :=
  calc
    w = ((3 * w + 1) / 3) - (1 / 3) := by ring
    _ = (4 / 3) - 1 / 3 := by rw [h1]
    _ = 1 := by ring
\end{lstlisting}

\section*{Exercise 4}
Link to code on \href{https://github.com/lucastassis/BU-CS511/HW01/code/HW01.lean}{GitHub}. The solution was: 

\begin{lstlisting}
-- Example 1.3.9
example {a b : ℚ} (h1 : a - 3 = 2 * b) : a ^ 2 - a + 3 = 4 * b ^ 2 + 10 * b + 9 :=
  calc
    a ^ 2 - a + 3 = ((a - 3) ^ 2 + 6 * a - 9) - a + 3 := by ring
    _ = (a - 3) ^ 2 + 5 * a - 6 := by ring
    _ = (a - 3) ^ 2 + 5 * ((a - 3) + 3) - 6 := by ring
    _ = (a - 3) ^ 2 + 5 * (a - 3) + 9 := by ring
    _ = (2 * b) ^ 2 + 5 * (2 * b) + 9 := by rw [h1]
    _ = 4 * b ^ 2 + 10 * b + 9 := by ring
\end{lstlisting}


\end{document}
