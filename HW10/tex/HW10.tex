\documentclass[11pt]{article}
\usepackage{xspace,epsfig,amsmath,amsthm,amssymb,fullpage}
\topmargin   0pt
\marginparwidth 0pt
\oddsidemargin  0pt
\evensidemargin 0pt
\marginparsep 0pt
\textwidth   6.5in
\textheight  9.2in

\usepackage{tikz}
\usepackage{tikz-qtree,tikz-qtree-compat} % for parse trees

\usepackage{latexsym}
\usepackage{amssymb}
\usepackage{amsmath}
\usepackage{amsthm}
\usepackage{amsfonts}
\usepackage{bm} 
% \usepackage{prooftree}
\usepackage{flagderiv}
\usepackage{logicproof}
\usepackage{bussproofs}
\usepackage{hyperref}
\usepackage{color}
\usepackage{listings}
\usepackage{synttree} 
\usepackage{pdflscape}
\usepackage{enumerate}   

\definecolor{keywordcolor}{rgb}{0.7, 0.1, 0.1}   % red
\definecolor{tacticcolor}{rgb}{0.0, 0.1, 0.6}    % blue
\definecolor{commentcolor}{rgb}{0.4, 0.4, 0.4}   % grey
\definecolor{symbolcolor}{rgb}{0.0, 0.1, 0.6}    % blue
\definecolor{sortcolor}{rgb}{0.1, 0.5, 0.1}      % green
\definecolor{attributecolor}{rgb}{0.7, 0.1, 0.1} % red
\newcommand*{\defeq}{\stackrel{\text{def}}{=}}

\def\lstlanguagefiles{lstlean.tex}
% set default language

\hypersetup{colorlinks=true,linkcolor=blue,urlcolor=blue}
% WARNING:
% Do NOT use package `bussproofs' and package `prooftree' at the same time,
% \begin{prooftree} ... \end{prooftree} is an environment defined in the
% package  `bussproofs', which conflicts with the name of the package
% `prooftree'.

\newcommand{\defn}{\overset{\text{\scriptsize def}}{=}}
\newcommand{\Intro}[1]{{#1}{\text{i}}}
\newcommand{\IntroA}[1]{{#1}{\text{i}_1}}
\newcommand{\IntroB}[1]{{#1}{\text{i}_2}}
\newcommand{\Elim}[1]{{#1}{\text{e}}}
\newcommand{\ElimA}[1]{{#1}{\text{e}_1}}
\newcommand{\ElimB}[1]{{#1}{\text{e}_2}}
\newcommand{\Set}[1]{\{ #1 \}}
\newcommand{\SET}[1]{\Bigl\{ #1 \Bigr\}}
\newcommand{\TTT}{\bm{\mathsf{T}}}
\newcommand{\FFF}{\bm{\mathsf{F}}}
\setlength{\parindent}{0pt}
% parameters: four corners, title, scribes' names
\newcommand{\handout}[6]{{
\begin{center}
\begin{minipage}{14cm}
    \setlength{\parindent}{0cm}%
    \fbox{\vbox{%
        {#1}%
        \hfill
        #2

        \center{\Large\bf{#5}}

        \emph{#3}\hfill #4
    }}%
    \vskip0pt
    \vbox{\hfill  #6}%
\end{minipage}
\\ 
\end{center}
}}

%%%%%%%%%%%%%%%%%%%%%%%%%%%%%%%%%%%%%%%%%%%%%%%%%%%%%%%%%%%%%%%%%%%%%%

\begin{document}
\handout{CS 511 Formal Methods, Fall 2024}
        {Instructor: Assaf Kfoury}
        {November 14, 2024}
        {Lucas Miguel Tassis}
        {Homework Assignment 10}

%%  THE FOLLOWING IS USED WITH THE logicproof ENVIRONMENT:
%%
\setlength{\subproofhorizspace}{.5em}
\setlength{\intersubproofvertspace}{0.5em}
\lstset{language=lean}


\section*{Exercise 1} 
Yes, $\textrm{Th}(\mathcal{M}) = \{ \varphi \mid \varphi \textrm{ is a first-order sentence s.t } \mathcal{M} \models \varphi\}$ is deductively closed. Since we are considering first-order logic, then we have that $\mathcal{M} \models \varphi \to \ \mathcal{M} \vdash \varphi$ by \textit{completeness}. Therefore, we can say that $\textrm{Th}(\mathcal{M})$ is deductively closed, because for every $\mathcal{M} \models \varphi$, we also have $\mathcal{M} \vdash \varphi$ by using completeness.

\section*{Exercise 2}

\begin{enumerate}
    \item The first-order sentence will be defined as:
    
    $$\varphi_1 \defeq \forall x (B(x) \lor G(x) \lor P(x) \lor Y(x))$$
    
    \item The first-order sentence will be defined as:
    
    \begin{align*}
        \varphi_2 \defeq \forall x \Big( & \big(B(x) \land \lnot G(x) \land \lnot P(x) \land \lnot Y(x)\big) \lor \\
          & \big(\lnot B(x) \land G(x) \land \lnot P(x) \land \lnot Y(x)\big) \lor\\
          &\big(\lnot B(x) \land \lnot G(x) \land P(x) \land \lnot Y(x)\big) \lor \\
          &\big(\lnot B(x) \land \lnot G(x) \land \lnot P(x) \land Y(x)\big)\Big)
    \end{align*}

    \item The first-order sentence will be defined as:
    
    \begin{align*}
        \varphi_3 \defeq \forall x \forall y \Big( \big(\lnot (x \approx y) \land R(x, y)\big) \to \big[ & \big(B(x) \land (G(y) \lor P(y) \lor Y(y))\big) \lor \\
        & \big(G(x) \land (B(y) \lor P(y) \lor Y(y))\big) \lor \\
        & \big(P(x) \land (B(y) \lor G(y) \lor Y(y))\big) \lor \\
        & \big(Y(x) \land (B(y) \lor G(y) \lor P(y))\big)\big]\Big)
    \end{align*}
    
    \item Considering that $\mathcal{M}$ is an infinite planar graph, we have from Hint 2 that every finite subgraph of $\mathcal{M}$ is also planar. Additionally, we have from Hint 1, that every finite planar graph is four-colorable. Thus, since $\mathcal{M}$ is finitely satisfiable (every finite subgraph has a four-coloring from Hints 1 and 2), then $\mathcal{M}$ must also be four-colorable by \textit{compactness}. 
\end{enumerate}


\section*{Exercise 3}
The Lean template file with the solutions is available on \href{https://github.com/lucastassis/BU-CS511/blob/main/HW10/code/HW10.lean}{GitHub}.

\section*{Exercise 4}
The Lean template file with the solutions is available on \href{https://github.com/lucastassis/BU-CS511/blob/main/HW10/code/HW10.lean}{GitHub}.

\section*{Problem 2}
The Lean template file with the solutions is available on \href{https://github.com/lucastassis/BU-CS511/blob/main/HW10/code/HW10.lean}{GitHub}.




\end{document}
