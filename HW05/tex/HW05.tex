\documentclass[11pt]{article}
\usepackage{xspace,epsfig,amsmath,amsthm,amssymb,fullpage}
\topmargin   0pt
\marginparwidth 0pt
\oddsidemargin  0pt
\evensidemargin 0pt
\marginparsep 0pt
\textwidth   6.5in
\textheight  9.2in

\usepackage{tikz}
\usepackage{tikz-qtree,tikz-qtree-compat} % for parse trees

\usepackage{latexsym}
\usepackage{amssymb}
\usepackage{amsmath}
\usepackage{amsthm}
\usepackage{amsfonts}
\usepackage{bm} 
% \usepackage{prooftree}
\usepackage{flagderiv}
\usepackage{logicproof}
\usepackage{bussproofs}
\usepackage{hyperref}
\usepackage{color}
\usepackage{listings}
\usepackage{synttree} 

\definecolor{keywordcolor}{rgb}{0.7, 0.1, 0.1}   % red
\definecolor{tacticcolor}{rgb}{0.0, 0.1, 0.6}    % blue
\definecolor{commentcolor}{rgb}{0.4, 0.4, 0.4}   % grey
\definecolor{symbolcolor}{rgb}{0.0, 0.1, 0.6}    % blue
\definecolor{sortcolor}{rgb}{0.1, 0.5, 0.1}      % green
\definecolor{attributecolor}{rgb}{0.7, 0.1, 0.1} % red

\def\lstlanguagefiles{lstlean.tex}
% set default language

\hypersetup{colorlinks=true,linkcolor=blue,urlcolor=blue}
% WARNING:
% Do NOT use package `bussproofs' and package `prooftree' at the same time,
% \begin{prooftree} ... \end{prooftree} is an environment defined in the
% package  `bussproofs', which conflicts with the name of the package
% `prooftree'.

\newcommand{\defn}{\overset{\text{\scriptsize def}}{=}}
\newcommand{\Intro}[1]{{#1}{\text{i}}}
\newcommand{\IntroA}[1]{{#1}{\text{i}_1}}
\newcommand{\IntroB}[1]{{#1}{\text{i}_2}}
\newcommand{\Elim}[1]{{#1}{\text{e}}}
\newcommand{\ElimA}[1]{{#1}{\text{e}_1}}
\newcommand{\ElimB}[1]{{#1}{\text{e}_2}}
\newcommand{\Set}[1]{\{ #1 \}}
\newcommand{\SET}[1]{\Bigl\{ #1 \Bigr\}}
\newcommand{\TTT}{\bm{\mathsf{T}}}
\newcommand{\FFF}{\bm{\mathsf{F}}}
\setlength{\parindent}{0pt}
% parameters: four corners, title, scribes' names
\newcommand{\handout}[6]{{
\begin{center}
\begin{minipage}{14cm}
    \setlength{\parindent}{0cm}%
    \fbox{\vbox{%
        {#1}%
        \hfill
        #2

        \center{\Large\bf{#5}}

        \emph{#3}\hfill #4
    }}%
    \vskip0pt
    \vbox{\hfill  #6}%
\end{minipage}
\\ 
\end{center}
}}

%%%%%%%%%%%%%%%%%%%%%%%%%%%%%%%%%%%%%%%%%%%%%%%%%%%%%%%%%%%%%%%%%%%%%%

\begin{document}
\handout{CS 511 Formal Methods, Fall 2024}
        {Instructor: Assaf Kfoury}
        {October 10, 2024}
        {Lucas Miguel Tassis}
        {Homework Assignment 5}

%%  THE FOLLOWING IS USED WITH THE logicproof ENVIRONMENT:
%%
\setlength{\subproofhorizspace}{.5em}
\setlength{\intersubproofvertspace}{0.5em}
\lstset{language=lean}


\section*{Exercise 1} 

\subsection*{(a) $(y \approx 0) \land (y \approx x) \vdash 0 \approx x$}

\begin{logicproof}{3}
    (y \approx 0) \land (y \approx x) & premise \\
    y \approx 0 & $\Elim{\land}$ 1 \\
    y \approx x & $\Elim{\land}$ 1 \\
    0 \approx x & $\Elim{=}$ 2, 3
\end{logicproof}
where, $t_1$ is $y$, $t_2$ is 0, and $\phi(y)$ is $y \approx x$


\subsection*{(b) $t_1 \approx t_2 \vdash (t + t_2) \approx (t + t_1)$}

\begin{logicproof}{3}
    t_1 \approx t_2 & premise \\
    t + t_1 \approx t + t_1 & $\Intro{=}$ \\
    t + t_2 \approx t + t_1 & $\Elim{=}$ 1, 2
\end{logicproof}
where, $t_1$ is $t_1$, $t_2$ is $t_2$, and $\phi(x)$ is $t + x \approx t + t_2$

\section*{Exercise 2}

\subsection*{(a) $\exists x (S \to Q(x)) \vdash S \to \exists x Q(x)$}

\begin{logicproof}{3}
    \exists x (S \to Q(x)) & premise \\
    \begin{subproof}
        S & assumption \\
        \begin{subproof}
            x_0 & fresh $x_0$ \\
            S \to Q(x_0) & assumption \\
            Q(x_0) & $\Elim{\to}$ 2, 3 \\
            \exists Q(x) & $\Intro{\exists}$ 4
        \end{subproof}
        \exists Q(x) & $\Elim{\exists}$ 1, 3--5
    \end{subproof}
    S \to \exists x Q(x) & $\Intro{\to}$ 2--6
\end{logicproof}

\subsection*{(b) $\forall x P(x) \to S \vdash \exists x (P(x) \to S)$}

There are two solutions for this exercise, one using the identities and equivalences, and one using only natural deduction proof. I will show both, but the pure natural deduction proof was based on web search. The first one using some identities and equivalences is:

\begin{logicproof}{4}
    \forall x P(x) \to S & premise \\
    \lnot \forall x P(x) \lor S & 1, \href{https://en.wikipedia.org/wiki/Logical_equivalence#Logical_equivalences_involving_conditional_statements}{equivalence 3} (can be proved using truth table) \\
    \exists x \lnot P(x) \lor S & 2, \href{https://en.wikipedia.org/wiki/First-order_logic#Provable_identities}{provable identity 1} \\
    \exists x (\lnot P(x) \lor S) & 3, \href{https://en.wikipedia.org/wiki/Prenex_normal_form#Conjunction_and_disjunction}{prenex normal form} (since $x$ is not free on $S$ we do not need to rename variables) \\
    \exists x (P(x) \to S) & 4, \href{https://en.wikipedia.org/wiki/Logical_equivalence#Logical_equivalences_involving_conditional_statements}{equivalence 3} (can be proved using truth table)
\end{logicproof}

The second one is using only natural deduction (based on web search):

\begin{logicproof}{4}
    \forall x P(x) \to S & premise \\
    \begin{subproof}
        \lnot \exists x (P(x) \to S) & assumption \\
        \begin{subproof}
            x_0 & fresh $x_0$\\
            \begin{subproof}
                \lnot P(x_0) & assumption \\
                \begin{subproof}
                    P(x_0) & assumption \\
                    \bot & $\Elim{\lnot}$ 4, 5 \\
                    S & $\Elim{\bot}$ 6
                \end{subproof}
                P(x_0) \to S & $\Intro{\to}$ 5--7 \\
                \exists x (P(x) \to S) & $\Intro{\exists x}$ 8 \\
                \bot & $\Elim{\lnot}$ 2, 9
            \end{subproof}
            \lnot \lnot P(x_0) & $\Intro{\lnot}$ 4--10 \\
            P(x_0) & $\Elim{\lnot \lnot }$ 11
        \end{subproof}
        \forall x P(x) & $\Intro{\forall x}$ 3--12 \\
        S & $\Elim{\to}$ 1, 13 \\
        \begin{subproof}
            P(t) & assumption \\
            S & copy 14
        \end{subproof}
        P(t) \to S & $\Intro{\to}$ 15--16 \\
        \exists x (P(x) \to S) & $\Intro{\exists x}$ 17 \\
        \bot & $\Elim{\lnot}$ 2, 18
    \end{subproof}
    \lnot \lnot \exists x (P(x) \to S) & $\Intro{\lnot}$ 2--19 \\
    \exists x (P(x) \to S) & $\Elim{\lnot \lnot}$ 20
    
\end{logicproof}


\section*{Exercise 3}
The Lean template file with the solutions is available on \href{https://github.com/lucastassis/BU-CS511/blob/main/HW05/code/HW05.lean}{GitHub}.

\section*{Exercise 4}
The Lean template file with the solutions is available on \href{https://github.com/lucastassis/BU-CS511/blob/main/HW05/code/HW05.lean}{GitHub}.




\end{document}
